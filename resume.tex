%%%%%%%%%%%%%%%%%%%%%%%%%%%%%%%%%%%%%%%%%%%%%%%%%%%%%%%%%%%%%%%%%%%%%%%%%%%%%%%%
% Medium Length Graduate Curriculum Vitae
% LaTeX Template
% Version 1.2 (3/28/15)
%
% This template has been downloaded from:
% http://www.LaTeXTemplates.com
%
% Original author:
% Rensselaer Polytechnic Institute 
% (http://www.rpi.edu/dept/arc/training/latex/resumes/)
%
% Modified by:
% Daniel L Marks <xleafr@gmail.com> 3/28/2015
% 
% Further modified by:
% Rohan Bavishi <rohan.bavishi95@gmail.com> 9/20/2016
%
% Used by:
% Penserbjorne  <paul.aguilar.enriquez@hotmail.com> 29/11/2016
%
% Important note:
% This template requires the simple_style.cls file to be in the same directory 
% as the .tex file. The res.cls file provides the resume style used for 
% structuring the document.
%
%%%%%%%%%%%%%%%%%%%%%%%%%%%%%%%%%%%%%%%%%%%%%%%%%%%%%%%%%%%%%%%%%%%%%%%%%%%%%%%%

%-------------------------------------------------------------------------------
%	PACKAGES AND OTHER DOCUMENT CONFIGURATIONS
%-------------------------------------------------------------------------------

%%%%%%%%%%%%%%%%%%%%%%%%%%%%%%%%%%%%%%%%%%%%%%%%%%%%%%%%%%%%%%%%%%%%%%%%%%%%%%%%
% You can have multiple style options the legal options ones are:
%
%   centered:	the name and address are centered at the top of the page 
%				(default)
%
%   line:		the name is the left with a horizontal line then the address to
%				the right
%
%   overlapped:	the section titles overlap the body text (default)
%
%   margin:		the section titles are to the left of the body text
%		
%   11pt:		use 11 point fonts instead of 10 point fonts
%
%   12pt:		use 12 point fonts instead of 10 point fonts
%
%%%%%%%%%%%%%%%%%%%%%%%%%%%%%%%%%%%%%%%%%%%%%%%%%%%%%%%%%%%%%%%%%%%%%%%%%%%%%%%%

\documentclass[mm]{simple_style}  

\usepackage[spanish]{babel}
% Default font is the helvetica postscript font
\usepackage{helvet}
\usepackage{hyperref}
\usepackage{url}
\usepackage{xcolor}
\hypersetup {
    colorlinks=true,
    linkcolor=colorlink,
    filecolor=magenta,      
    urlcolor=colorlink,
}
\usepackage[left=0.7in, right=2in, top=0.9in]{geometry}

% Increase text height
\textheight=700pt

\begin{document}

%-------------------------------------------------------------------------------
%	NAME AND ADDRESS SECTION
%-------------------------------------------------------------------------------
\name{Paul Sebastian Aguilar Enriquez}
\qualification{Estudiante de Ingenier\'ia en Computaci\'on}
\phoneone{+52 1 3312438124}
\phonetwo{+52 1 5545891761}
\emailone{paul.aguilar.enriquez@hotmail.com}
%\emailtwo{}
%\website{http://}{\url{}}
\github{https://github.com/penserjorne}{\url{github.com/penserbjorne}}
\linkedin{https://mx.linkedin.com/in/penserbjorne}{\url{mx.linkedin.com/in/penserbjorne}}

\address{Nustepec \# 19, CP 04366\\Pedregal de Santo Domingo\\Delegaci\'on Coyoac\'an\\Ciudad de M\'exico, M\'exico}
%-------------------------------------------------------------------------------

\begin{resume}

%-------------------------------------------------------------------------------
%	ALGO ACERCA DE MI
%-------------------------------------------------------------------------------
\section{Datos Personales}
Estudiante de Ingenier\'ia en Computaci\'on en la Facultad de Ingenier\'ia de la Universidad Nacional Aut\'onoma de M\'exico.\\
Originario de Zapopan, Jalisco, actualmente resido en la Ciudad de M\'exico.\\
Fecha de nacimiento: 16 de febrero de 1994\\
\sectionline
%-------------------------------------------------------------------------------

%-------------------------------------------------------------------------------
%	EDUACION
%-------------------------------------------------------------------------------
\section{Educaci\'on}
\cusemph{Universidad Nacional Aut\'onoma de M\'exico}, Ciudad de M\'exico, M\'exico\\
{\sl Ingenier\'ia en Computaci\'on}, Facultad de Ingenier\'ia. \timeline{Agosto 2014 - Actualidad}\\
%\cusemph{GPA: 9.0/10} (Overall)\\
\\
\cusemph{Centro de Ense\~nanza T\'ecnica Industrial}, Guadalajara, Jalisco, M\'exico\\
{\sl Tecn\'ologo en Inform\'atica y Computaci\'on, m\'odulo de ''Software''}\\
Plantel Colomos. \timeline{Agosto 2009 - Julio 2013}\\
%\cusemph{GPA: 9.0/10} (Overall)\\
\\
\sectionline
%-------------------------------------------------------------------------------

%-------------------------------------------------------------------------------
%	EXPERIENCIA LABORAL
%-------------------------------------------------------------------------------
\section{Experiencia\\Laboral}
\begin{project}
\title{Software de Traceabilidad de Proyectos}
\supervisor{AVALTO - Ciudad De M\'exico / Guadalajara , M\'exico}
\duration{Enero 2016 - Marzo 2016}
\description{
	- Desarrollo del sistema web STP (Software de Traceabilidad de Proyectos) para la empresa EFIRSA.
}
\end{project}
\begin{project}
\title{M\'odulos para el sistema scada SIMCOT}
\supervisor{SICCOA - Ciudad de M\'exico, M\'exico}
\duration{Marzo 2015 - Julio 2015}
\description{
	- Realizaci\'on de dos m\'odulos para el sistema scada SIMCOT de PEMEX.\\
	- El primer m\'odulo se realiz\'o en C++ con Visual Studio 6.0 para Windows 98. El m\'odulo se desarroll\'o para la planta ''Pajaritos'' de Coatzacoalcos, Veracruz.\\
	- El segundo m\'odulo se realiz\'o en C++ con Visual Studio 2005 para Windows Server 2003. El m\'odulo se desarroll\'o para la planta de L\'azaro C\'ardenas, Michoac\'an de Ocampo.\\
	- En ambos m\'odulos se migro el uso de INFORMIX a MySQL.
}
\end{project}
\begin{project}
\title{Sistema DYSIGA}
\supervisor{DesarrolloSoft - Zapopan, Jalisco, M\'exico}
\duration{Enero 2014 - Julio 2014}
\description{
	- Desarrollo de los m\'odulos de ''Administraci\'on'' y ''Ventas'' para el sistema DYSIGA de la constructora San Lorenzo.\\
	- Los m\'odulos fueron desarrollad\'os en WPF 4.5 con Visual Studio 2013.
}
\end{project}
\begin{project}
\title{Empleado en fonda}
\supervisor{El sazon de mam\'a - Guadalajara, Jalisco, M\'exico}
\duration{Diciembre 2013 - Marzo 2014}
\description{
	- Realice todas las actividades necesarias en el establecimiento excepto cocinar.\\
	- Lavaplatos, aguador, mesero, repartidor, intendente.
}
\end{project}
\begin{project}
\title{Sistema RITAM}
\supervisor{SI$T^2$ - Zapopan, Jalisco, M\'exico}
\duration{Septiembre 2013 - Diciembre 2013}
\description{
	- Desarrollo web para el sistema RITAM de la empresa IMI.\\
	- El desarrollo incluyo configuraci\'on y manejo de un servidor SQL Server2008R2.\\
	- Se desarroll\'o un m\'odulo en Visual Basic 6.0 para bloqueo total de equipos y activaci\'on de una alarma para las estaciones de trabajo.
}
\end{project}
\vspace{-2ex}
\sectionline
%-------------------------------------------------------------------------------

%-------------------------------------------------------------------------------
%	IDIOMAS
%-------------------------------------------------------------------------------
\section{Idiomas}
\cusemph{Espa\~nol}: Nativo $|$ \cusemph{Ingl\'es}: Intermedio $|$ \cusemph{Portugu\'es}: B\'asico $|$ \cusemph{Nahuatl}: B\'asico

\vspace{-2ex}
\sectionline
%-------------------------------------------------------------------------------

%-------------------------------------------------------------------------------
%	CONOCIMIENTOS TECNICOS
%-------------------------------------------------------------------------------
\section{Conocimientos\\T\'ecnicos}
\cusemph{Lenguajes}: C, C++, C\#, Visual Basic, Java, SQL, Ensamblador (x86, microcontroladores), VHDL, PHP, Javascript, HTML, \LaTeX.
\\
\cusemph{Sistemas Operativos}: Windows 98 a Windows 10, GNU/Linux, Firefox OS.
\\
\cusemph{Plataformas/Software}: git, Visual Studio/.NET, Netbeans, QT Creator, Unreal Engine, Unity, WAMP, Windows Server, Apache, Arduino, MS Office, WPS.
\\
\cusemph{Otros}: Conocimientos en electr\'onica (anal\'ogica, digital, de potencia, microcontroladores, microprocesadores, PLD's).
\\
Conocimientos b\'asicos de redes y administraci\'on de servidores.

\vspace{-2ex}
\sectionline
%-------------------------------------------------------------------------------

%-------------------------------------------------------------------------------
%	PROYECTOS PERSONALES
%-------------------------------------------------------------------------------
\section{Proyectos\\Personales}
\begin{project}
\title{Laboratorio de Investigaci\'on y Desarrollo de Software Libre (LIDSOL)}
\supervisor{Encargado del LIDSOL}
\duration{Noviembre 2016 - Actualidad}
\description{
Encargado de la reestructuraci\'on y reapertura del Laboratorio de Investigaci\'on y Desarrollo de Software Libre (LIDSOL) de la Facultad de Ingenier\'ia.
}
\end{project}
\begin{project}
\title{MexMod}
\supervisor{Roms de Firefox OS}
\duration{Diciembre 2014 - Abril 2016}
\description{	
Compilaba las fuentes de Firefox OS y liberaba las im\'agenes resultantes para ciertas terminales.\\
\href{http://mexmod.com/}{Sitio de MexMod}
}
\end{project}\vspace{-2ex}
\sectionline
%-------------------------------------------------------------------------------

%-------------------------------------------------------------------------------
%	VOLUNTARIADO
%-------------------------------------------------------------------------------
\section{Voluntariado}
\begin{project}
\title{Mozillian en Mozilla Mexico}
\supervisor{Voluntario en Mozilla M\'exico}
\duration{Marzo 2015 - Actualidad}
\description{
Contribuyo en la comunidad de Mozilla M\'exico apoyando en la secci\'on general, ya sea con material interno de la comunidad o eventos.
}
\end{project}
\begin{project}
\title{escuelaIT - CongresoM\'ovilApp2015}
\supervisor{Charla "Mi primera aplicaci\'on en Firefox OS"}
\duration{Septiembre 2015}
\description{	
En esta charla se explicaron desde conceptos b\'asicos del sistema operativo, su arquitectura, las APIs disponibles, el entorno de trabajo para desarrollar una app, como "localizar" tu app y como subirla al marketplace.
}
\end{project}
\begin{project}
\title{Mozilla M\'exico}
\supervisor{Campus Party M\'exico 2015 - CPMX6}
\duration{Julio 2015}
\description{	
Asistente general por parte de la comunidad de Mozilla M\'exico.\\
Di una peque\~na charla para las comunidades de mi experiencia en Mozilla, el proceso de compilaci\'on de una rom para Firefox OS y el desarrollo de apps para el mismo.
}
\end{project}
\begin{project}
\title{Hack CDMX}
\supervisor{Voluntario en el Hack CDMX 2015.}
\duration{Marzo 2015}
\description{	}
\end{project}
\begin{project}
\title{Programa Universitario de Bio\'etica, UNAM}
\supervisor{Voluntario en la campa\~na ''Velo en Perspectiva''}
\duration{Octubre 2014}
\description{	
El enfoque de esta campa\~na fue el realizar consciencia sobre el plagio.}
\end{project}
\begin{project}
\title{UNAM Mobile}
\supervisor{Voluntario del Congreso Universitario M\'ovil y del HACK UNAM 2014.}
\duration{Septiembre 2014}
\description{	}
\end{project}
\vspace{-2ex}
\sectionline
%-------------------------------------------------------------------------------

%-------------------------------------------------------------------------------
% PREMIOS
%-------------------------------------------------------------------------------
\section{Concursos}
\begin{project}
\title{Hackaton Idea Camp 2015}
\supervisor{Reto de MexBT}
\duration{Diciembre 2015}
\description{
 - Ganador del reto de MexBT en el ''Hackaton Idea Camp 2015''.\\
 - Se realizo un sistema para mover remesas de Estados Unidos a M\'exico haciendo uso de la API de MexBT.\\
 - El sistema constaba de dos clientes, un cliente de escritorio desarrollado en WPF/C\# con conexi\'on a MySQL que implementaba la API de MexBT y un cliente web para consulta de datos.
}
\end{project}
\begin{project}
\title{Hack For Equality}
\supervisor{Mejor implementaci\'on t\'ecnica}
\duration{Febrero 2016}
\description{	
 - Ganador de "Mejor implementaci\'on t\'ecnica" (3er lugar) en Hack For Equality en la EBC.\\
 - El proyecto consisti\'o en una app m\'ovil para Firefox OS que permite realizar denuncias an\'onimas que se ven reflejadas en un sitio a trav\'es de mapas (cartoDB).\\
 - El sistema desarrollado fue nombrado TakPod (denuncia en Maya).
}
\end{project}
\vspace{-2ex}
\sectionline
%-------------------------------------------------------------------------------

%-------------------------------------------------------------------------------
%	Hobbies
%-------------------------------------------------------------------------------
\section{Pasatiempos}
Software Libre, Cine, C\'omics, Series, Ciencia Ficci\'on, Anime, Manga, Videojuegos, Basketball, Actividades Recreativas, Comer.
%-------------------------------------------------------------------------------
\end{resume}
\end{document}
